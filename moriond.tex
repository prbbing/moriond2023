%====================================================================%
%                  MORIOND.TEX                                       %
%====================================================================%

\documentclass{moriond}

\usepackage{subcaption}
\captionsetup{compatibility=false}

\bibliographystyle{unsrt}    
% for BibTeX - sorted numerical labels by order of
% first citation.

% A useful Journal macro
\def\Journal#1#2#3#4{{#1} {\bf #2}, #3 (#4)}

% Some useful journal names
\def\NCA{\em Nuovo Cimento}
\def\NIM{\em Nucl. Instrum. Methods}
\def\NIMA{{\em Nucl. Instrum. Methods} A}
\def\NPB{{\em Nucl. Phys.} B}
\def\PLB{{\em Phys. Lett.}  B}
\def\PRL{\em Phys. Rev. Lett.}
\def\PRD{{\em Phys. Rev.} D}
\def\ZPC{{\em Z. Phys.} C}

% Some other macros used in the sample text
\def\st{\scriptstyle}
\def\sst{\scriptscriptstyle}
\def\mco{\multicolumn}
\def\epp{\epsilon^{\prime}}
\def\vep{\varepsilon}
\def\ra{\rightarrow}
\def\ppg{\pi^+\pi^-\gamma}
\def\vp{{\bf p}}
\def\ko{K^0}
\def\kb{\bar{K^0}}
\def\al{\alpha}
\def\ab{\bar{\alpha}}
\def\be{\begin{equation}}
\def\ee{\end{equation}}
\def\bea{\begin{eqnarray}}
\def\eea{\end{eqnarray}}
\def\CPbar{\hbox{{\rm CP}\hskip-1.80em{/}}}
%temp replacement due to no font
%%%%%%%%%%%%%%%%%%%%%%%%%%%%%%%%%%%%%%%%%%%%%%%%%%
%                                                %
%    BEGINNING OF TEXT                           %
%                                                %
%%%%%%%%%%%%%%%%%%%%%%%%%%%%%%%%%%%%%%%%%%%%%%%%%%

\newcommand{\Photo}{\includegraphics[height=35mm]{mypicture}}
%\newcommand{\Photo}{}

\begin{document}
\title{Highlights of ATLAS Search Results}

\author{Bingxuan Liu, on behalf of the ATLAS Collaboration}

\address{Department of Physics, Simon Fraser University, Vancouver, Canada}

\maketitle\abstracts{Searching for beyond standard model (BSM) physics has been
one of the primary goals of the Large Hadron Collider (LHC) physics program.
The LHC delivered 140 $\mathrm{fb}^{-1}$ data of high quality in Run 2,
allowing both the ATLAS and CMS collaborations to expand and improve their
search programs. The recent development in detector perforamnce and analysis
techniques have brought significant boosts to the search sensitivities. In this
article, highlights of recent ATLAS search results are discussed and
summarized.}  

\section{Introduction}

Many mysteries in particle physics, such as the hierarchy problem, the origin
of dark matter (DM) and neutrino masses are still waiting for answers or hints
from the Large Hadron Collider (LHC). ATLAS is a general purpose detector at
the LHC that is capable of searching for beyond standard model (BSM) physics
via various approaches. ATLAS has conducted a comprehensive set of searches in
the past years, excluding the majority part of the phase space. As a
consequence, recent ATLAS searches are featured with applying cutting-edge
analysis techniques or detector performance development, filling the gaps
between explored regions of phase space and considering challenging, completely
uncovered signatures.      

\section{Full Run 2 Upgrades}

The detector performance in ATLAS is continuously improving, thanks to the
diligent work carried out in the relevant areas. For instance, the performance
of bottom- and top-tagging has advance significantly in the past few years. In
addition, the application of machine learning techniques have become very
mature in physics analyses, enhancing the sensitivities further. Even though
previous analyses have explored similar final states, taking advantage of the
above facts, the full Run 2 upgrades of the those searches will push the
exclusion limits further.\\

In the full Run 2 ATLAS vector-like quark search, the top partner masses with
50\% decay width are excluded up to 1975 GeV for the singlet representation,
considering $B(T\rightarrow Wb)=0.5$ (~\ref{fig:limits}~\subref{fig:vlq}).
Comared to the previous analysis probing the same final state, this analysis
adopted an updated top-tagger and more optimized selections.        

\begin{figure}[htbp!]
     \centering
     \begin{subfigure}[b]{0.45\textwidth}
         \centering
         \includegraphics[width=\textwidth]{VLQ}
         \caption{}
         \label{fig:vlq}
     \end{subfigure}
     \begin{subfigure}[b]{0.45\textwidth}
         \centering
         \includegraphics[width=\textwidth]{RHN}
         \caption{}
         \label{fig:rhn}
     \end{subfigure}
        \caption{Three simple graphs}
        \label{fig:diagrams}
\end{figure}

\section{Filling the Gap}

There have been many BSM models proposed by the theory community in the past
decades. The number of free parameters in those models is so vastly large that
a single search can only probe a subset of the parameter space, a particular
combination of masses and couplings. Naturally, there are gap between exisitng
searches, and they need to be explored. The gap regions are usually hard to
probe so that special analysis strategies are essential.  


\begin{figure}[htbp!]
     \centering
     \begin{subfigure}[b]{0.32\textwidth}
         \centering
         \includegraphics[width=\textwidth]{bbyy}
         \caption{}
         \label{fig:vlq}
     \end{subfigure}
     \begin{subfigure}[b]{0.32\textwidth}
         \centering
         \includegraphics[width=\textwidth]{micro}
         \caption{}
         \label{fig:rhn}
     \end{subfigure}
     \begin{subfigure}[b]{0.32\textwidth}
         \centering
         \includegraphics[width=\textwidth]{excited}
         \caption{}
         \label{fig:rhn}
     \end{subfigure}
        \caption{Three simple graphs}
        \label{fig:diagrams}
\end{figure}

\section{Challenging Signatures}

\section*{References}

\begin{thebibliography}{99}
\bibitem{ja}C Jarlskog in {\em CP Violation}, ed. C Jarlskog
(World Scientific, Singapore, 1988).

\bibitem{ma}L. Maiani, \Journal{\PLB}{62}{183}{1976}.

\bibitem{bu}J.D. Bjorken and I. Dunietz, \Journal{\PRD}{36}{2109}{1987}.

\bibitem{bd}C.D. Buchanan {\it et al}, \Journal{\PRD}{45}{4088}{1992}.

\end{thebibliography}

\end{document}

%%%%%%%%%%%%%%%%%%%%%%
% End of moriond.tex  %
%%%%%%%%%%%%%%%%%%%%%%


%%% Local Variables: 
%%% mode: latex
%%% TeX-master: t
%%% End: 

%%% Local Variables: 
%%% mode: latex
%%% TeX-master: t
%%% End: 

%%% Local Variables: 
%%% mode: latex
%%% TeX-master: t
%%% End: 
